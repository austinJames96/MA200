\documentclass[12pt]{article}
\usepackage{hyperref, color, setspace}
\usepackage{amsmath}
\usepackage{amssymb}
\usepackage[margin=1.5in]{geometry}
\setlength{\parindent}{0cm}

\begin{document}
\begin{center}
{\textbf{LaTeX Paper - MA200}}

Due Thursday, Dec. 12

       


{\textbf{Austin Sypolt}}
\end{center}

\vspace{9pt}

 {\bf Theorem 4.8.1 (page 348)}

\bigskip

The general solution of a given {\bf linear first-order differential equation}\\ $y' + F(x)y = G(x)$ can be found by:
\bigskip

$y = [ e^{-\int F(x) dx} ] * [ \int G(x) * e^{\int F(x) dx } dx + C] $

\bigskip

(Assuming both F(x) and G(x) are continuous functions)


\bigskip
\bigskip

First we should understand the concepts and terms of what we are solving.

\bigskip
\bigskip

A {\bf first order differential equation} is an equation where there exists a function F(x,y) consisting of two variables (x and y) defined on a region in the x-y plane.

\bigskip
\bigskip

A {\bf linear differential equation} is a differential equation with a linear polynomial of the form:


\[a_0 (x)y + a_1 (x)y' + a_2 (x)y'' + ... + a_n (x)y^n + b(x) = 0 \]

\bigskip
\bigskip

 \[a_0 (x), ... , a_n (x), b(x) : \] \\(Arbitrary differentiable functions, these do not have to be linear)

\bigskip

\[y', ... , y^n : \] \\ (Successive derivates of the function y with respect to the variable x)

\bigskip
\bigskip

This is an {\bf ordinary differential equation (ODE)} meaning it contains one or more functions of an independent variable and its derivates. {\bf Partial differential equations} also exist however these are with respect to more than one independent variable and will not be addressed here.


Now that we understand what we're working with, let's get into the numbers, or letters... I guess.

\bigskip

$y' + F(x)y = G(x)$ (Here's our linear first order ODE) {\bf Equation 1}

\bigskip

We can use an {\bf integrating factor} to solve for our first order ODE. An integrating factor is a function that can be multipled by an ODE for the purpose of making it integrable.


The integrating factor that will be used for linear first order ODE will be:

\bigskip

$e^{\int F(x) dx}$


\bigskip

Multiply this factor by all our terms to give:

\bigskip

$y'e^{\int F(x) dx} + F(x)ye^{\int F(x) dx}  = G(x)e^{\int F(x) dx} $

\bigskip
\bigskip

That's a mess, so instead let's simplify our integrating factor:

\bigskip

I = $e^{\int F(x) dx}$


\bigskip
\bigskip

Now to clean everything up a little bit substitute I for our integrating factor from Equation 1. This is an {\bf exact differential equation}.

\bigskip
$y'I + F(x)yI  = G(x)I $

\bigskip
\bigskip

Both sides will be then integrated:

${\int [y'I + F(x)yI] dx} = {\int [G(x)I] dx}$

\bigskip

From the product rule we get $(yI)' = y'I + F(x)yI$ 

This is why our integrating factor was chosen as $e^{\int F(x) dx}$! It can now be integrated easily and rewritten as:

{\bf $yI = {\int [G(x)I]dx + C}$ } {\bf | Equation 2}


\bigskip

Wait a second, remember how I = $e^{\int F(x) dx}$ ?

\bigskip

So let's expand Equation 2 with our integrating factor.

\bigskip

$ye^{\int F(x) dx} = {\int [G(x)e^{\int F(x) dx}]dx + C}$ 

\bigskip

Divide both sides by the integrating factor:

$y = [ \int G(x) * e^{\int F(x) dx } dx + C] / [ e^{\int F(x) dx} ] $

\bigskip
\bigskip
\begin{center}
$y = [ e^{-\int F(x) dx} ] * [ \int G(x) * e^{\int F(x) dx } dx + C]$

\end{center}

\raggedleft $\blacksquare$


\end{document}
